\documentclass{report}
\usepackage{graphicx}
\newcommand{\bs}{$\backslash$}

                             % The preamble begins here.
\title{Another Signal Manager}
\author{Erwin Waterlander}
\date{Dec 3 2022}          % Deleting this command produces today's date.

\newcommand{\bc}{\scriptsize}
\newcommand{\ec}{\normalsize}

\begin{document}             % End of preamble and beginning of text.

\maketitle                   % Produces the title.

\tableofcontents

\chapter{Introduction}\label{chap:intro}

This document can be downloaded from this URL:\\
\texttt{https://waterlan.home.xs4all.nl/asm.html}


\section{What is ASM?}


ASM is a program for digital signal processing for educational
purposes.

This is a port of the original version made in 1993 by Edwin Zoer and me on an
Acorn Archimedes computer running RISC Os.

\section{History}

\subsection{AIM : Another Image Manager}

also known as:
Atari Image Manager,
Archimedes Image Manager,
Amiga Image Manager. 

\paragraph{}
The image processing program AIM was originally developed for the 
ATARI-ST  by Frans Groen and  Robert de Vries.  Since  the  first 
version  of  AIM, the improvement of  this  public  domain  image 
processing  package  has  become a joint effort of  a  number  of 
people from the Delft University of Technology and the University 
of  Amsterdam.
AIM has been ported to the ARCHIMEDES (Arthur version) by  Robert
Ellens, Damir Sudar and Alle-Jan van der Veen.
Ed Doppenberg  was successful in the port to RISC Os. 
AIM has been written in the C-language. 
AIM is limited in functionality as well as in flexibility.
The main  purpose of the program is to experiment with digital
image processing.

The latest version was 3.15 (1995).

On the Polytechnic of Enschede the Archimedes RISC Os version of AIM was used
in practical lessons in image processing.  Polytechnic of Enschede (Hogeschool
Enschede), the Netherlands, is called Saxion hogescholen (www.saxion.nl)
today.


\subsection{ASM : Another Signal Manager}

In 1993 the idea came to make a program like AIM, but then for signal
processing: ASM for RISC Os.  The task of our final examination for the
Polytechnic of Enschede was to create ASM for RISC Os.

We made ASM at and with support of the Technical University of Delft, faculty
Applied Physics, Pattern Recognition group (Tom Hoeksma), and with support from
the Technisch Physische Dienst, Delft (Ed Doppenberg).  Our starting point was
a stripped down version of AIM made by Ed Doppenberg.  It was only one window
with a command line interpreter. 

In 1993 Edwin and I had only basic knowledge of ANSI C and no knowledge about
making user interfaces for RISC Os. Our goal was to put as much as possible
functionality in the program in only three months.

With the first RISC Os version we created it was possible to generate signals
and do some basic processing on them. The program was made for use during
practical lessons in digital signal processing at the polytechnic in Enschede.

The original intention was that other students would develop ASM further, but
this never happened. It was Ed Doppenberg who did a thorough revision of the
source code and added some professional functionality. That version of ASM (for
RISC Os) is not free available for the public domain.

In 1997 I ported the original version of ASM for RISC Os to DOS using DJGPP
2.01 (gcc 2.7.2). I used the Allegro 2.2 graphics library.  Allegro is a
library intended for making computer games. It was initially conceived on the
Atari ST.

In 1997 my primary goal was to port the program to a working version on DOS, for
the fun of programming and because I had no Archimedes computer.  That means I
only changed the graphical interface. I tried to keep the source code as much
as possible the same.

Allegro development went on, supporting more platforms.  In 2007 I build ASM
for DOS, Windows, and Linux using a newer version of Allegro.  Some minor
problems were fixed. For the rest it was the same version as in 1997.

For many years I had in the back of my mind the idea to make a windowed
version, like the original version on RISC Os. In 2021 I started porting ASM to
Java and JavaFX. In Dec 2022 the port to Java was ready. It is practically the
same as the original version of 1993 with the addition of menus and dialogs for
most of the commands and several bugs fixed.

\paragraph{}
This version of ASM is Public Domain software.\\\\
Erwin Waterlander\\
e-mail: waterlan@xs4all.nl


\chapter{User interface}

\section{Graphical interface}

The main window is a console in which the user can type the
commands. Each signal is displayed in a separate window.

\section{Command line interface}

ASM has the same command line interface as AIM.

\paragraph{}
Commands can be abbreviated as long as they stay unique.

\paragraph{}
Parameters are separated by spaces.
If you don't give all the parameters that are possible on
a certain command ASM will take default values.

\paragraph{}
A single question mark `\textbf{?}' (without quotes) or
`\textbf{-h}' as argument will give a short help line.

\paragraph{}
A dot`\textbf{.}' as argument can be used to use the default value.

\chapter{Functionality}

\section{Data-format conversion} 

All data is in 64 bit floating point (type double).

\section{Domain conversion}

Conversions can be done between the different domains:

\begin{table}[h]
\begin{center}
\begin{tabular}{|l|c|c|c|c|c|}
\hline
From \bs To  &  Time & Frequency & Amplitude & Magnitude &   Phase \\
\hline
Time      &       &     X     &     X     &     X     &     X \\
\hline
Frequency &   X   &           &           &     X     &     X \\
\hline
Amplitude & & & & & \\
\hline
Magnitude & & & & & \\
\hline
Phase     & & & & & \\
\hline
\end{tabular}
\label{tab:conversions}
\caption{Domain conversions}
\end{center}
\end{table}

\subsection{From time to frequency}

\paragraph{fft}: Fast-Fourier- Transformation

\bc
\begin{verbatim}
command: fft        input-signal, output-signal, length, window-type, average-type
default  :              a       ,       b      ,   9   ,      1     ,      0
range    :            <a-z>     ,     <a-z>    , <7-12>,    <0-6>   ,    <0-1>
domain   : time
\end{verbatim}
\ec

\begin{verbatim}
Windows:
0  block
1  Hanning
2  Hamming
3  Gauss
4  Blackman
5  Kaiser
6  triangle
\end{verbatim}

average-type, see section \ref{sec:freqtomag} and \ref{sec:freqtophas}.

FFT is done per record. The default length is the record length of the input signal.

\subsection{From frequency to time}

\paragraph{ifft}: Inverse Fast-Fourier-Transformation

\bc
\begin{verbatim}
command: ifft  input-signal, output-signal
default  :          a      ,       b
range    :        <a-z>    ,     <a-z>
domain   : frequency
\end{verbatim}
\ec

\subsection{From time to amplitude}

\paragraph{histogram}

\bc
\begin{verbatim}
command: histogram  input-signal, output-signal, buckets
default  :               a      ,       b       ,    9
range    :             <a-z>    ,     <a-z>     ,  <7-12>
domain   : time
\end{verbatim}
\ec

The number of buckets is given as a power of 2. So 9 means $2^9=512$ buckets.

\subsection{From frequency to magnitude}\label{sec:freqtomag}

There are two different ways of averaging:\\

Average-type 0:

\begin{enumerate}
\item Calculate magnitude for every record:\\
       \( |F(u)| = \sqrt{ Re^{2}(u) + Im^{2}(u) } , u = 0,1,...,N-1 \)
\item sum all the results from the different records
\item and calculate \( 10\cdot\log \).
\end{enumerate}

Average-type 1:

\begin{enumerate}
\item Sum all the different records,
\item calculate the magnitude of the result\\
   \( |F( u)| = \sqrt{ Re^{2}( u) + Im^{2}( u) } , u = 0,1,...,N-1 \)
\item and calculate \( 10\cdot\log \).
\end{enumerate}
\paragraph{magnitude}: Calculate the signal's magnitude

\bc
\begin{verbatim}
command: : magnitude  input-signal, output-signal, channel-no, average-type,  log
default  :                 a      ,  input-signal,     0     ,      0      ,   0
range    :               <a-z>    ,     <a-z>     ,  <0-max> ,    <0-1>    , <0-1>
domain   : frequency

log:
0 = linear Y-axis
1 = log Y- axis
\end{verbatim}
\ec

\subsection{From frequency to phase}\label{sec:freqtophas}

Average-type 0:

\begin{enumerate}
\item Calculate the phase of every record:\\
  \( \Phi(u) = tan^{-1}\left[ \frac{Im(u)}{Re(u)}\right],  u = 0, 1, ..., N-1\)
\item Sum the different records and divide by the number of records.
\end{enumerate}

Average-type 1:

\begin{enumerate}
\item Sum the different records and divide by the number of records.
\item Calculate the phase of the result:\\
   \( \Phi(u) = tan^{-1}\left[ \frac{Im(u)}{Re(u)}\right],  u = 0, 1, ..., N-1 \)
\end{enumerate}
\ec
\paragraph{phase}: Calculate the signal's phase

\bc
\begin{verbatim}
command: : phase      input-signal, output-signal, channel-no, average-type
default  :                 a      ,  input-signal,     0     ,      0
range    :               <a-z>    ,     <a-z>     ,  <0-max> ,    <0-1>
domain   : frequency
\end{verbatim}
\ec

\subsection{From time to magnitude}

Not implemented.

\subsection{From time to phase}

Not implemented.

\section{Mathematical functions}

There are a few simple mathematical functions in ASM. The functions
can be performed on signals in any domain. For functions that work
on more than one input-signal the record-length and the number of
channels have to be the same for all the input signals. All
constant values are values between $<min> ~ = INT\_MIN = -2147483647 $
and $ <max> ~ = INT\_MAX = 2147483647 $.

\paragraph{clear}: Make all elements 0

\[ Re(out) = 0 \]
\[ Im(out) = 0 \]

\bc
\begin{verbatim}
command: : clear      Input-signal
default  :                 a      
range    :               <a-z>    
domain   : time, frequency, amplitude, magnitude, phase
\end{verbatim}
\ec

\paragraph{assign}: Assign a constant value to all elements

\[ Re(out) = C \]
\[ Im(out) = C \]

\bc
\begin{verbatim}
command: : assign     Input-signal,   real-part,   imag-part
default  :                 a      ,       1    ,       1
range    :               <a-z>    ,   <min-max>,   <min-max>
domain   : time, frequency, amplitude, magnitude, phase
\end{verbatim}
\ec


\paragraph{inv}: Invert all elements

\[ Re(out) = -1*Re(in) \]
\[ Im(out) = -1*Im(in) \]

\bc
\begin{verbatim}
command: : inv        Input-signal,  Output-signal
default  :                 a      ,   input-signal  
range    :               <a-z>    ,      <a-z>
domain   : time, frequency, amplitude, magnitude, phase
\end{verbatim}
\ec

\paragraph{conjugate}: 

\[ Re(out) = Re(in) \]
\[ Im(out) = -1*Im(in) \]

\bc
\begin{verbatim}
command: : conjugate  Input-signal,  Output-signal
default  :                 a      ,   input-signal  
range    :               <a-z>    ,      <a-z>
domain   : time, frequency, amplitude, magnitude, phase
\end{verbatim}
\ec

\paragraph{cabs}: Calculate absolute value of each element

\[ Re(output) = \sqrt{Re^{2}(input) + Im^{2}(input)} \]
\[ Im(output) = 0 \]

\bc
\begin{verbatim}
command: : cabs       Input-signal,  Output-signal
default  :                 a      ,   input-signal  
range    :               <a-z>    ,      <a-z>
domain   : time, frequency, amplitude, magnitude, phase
\end{verbatim}
\ec

\paragraph{cadd}: Add a constant value

\[ Re(out) = Re(in) + C \]
\[ Im(out) = Im(in) \]

\bc
\begin{verbatim}
command: : cadd       Input-signal,  constant,  Output-signal
default  :                 a      ,      0   ,   input-signal  
range    :               <a-z>    , <min-max>,      <a-z>
domain   : time, frequency, amplitude, magnitude, phase
\end{verbatim}
\ec


\paragraph{cmultiply}: Multiply by a constant value

\[ Re(out) = C*Re(in) \]
\[ Im(out) = C*Im(in) \]

\bc
\begin{verbatim}
command: : cmultiply  Input-signal,  constant,  Output-signal
default  :                 a      ,      1   ,   input-signal  
range    :               <a-z>    , <min-max>,      <a-z>
domain   : time, frequency, amplitude, magnitude, phase
\end{verbatim}
\ec

\paragraph{cdivide}: Divide by a constant

\[ Re(out) = \frac{Re(in)}{C} \]
\[ Im(out) = \frac{Im(in)}{C} \]

\bc
\begin{verbatim}
command: : cdivide    Input-signal,  constant,  Output-signal
default  :                 a      ,      1   ,   input-signal  
range    :               <a-z>    ,  <1-max> ,      <a-z>
domain   : time, frequency, amplitude, magnitude, phase
\end{verbatim}
\ec


\paragraph{abs}: Absolute difference between two signals

\[ Re(out) = \sqrt{ (Re(in1) - Re(in2))^{2} + (Im(in1) - Im(in2))^{2} } \]
\[ Im(out) = 0 \]

\bc
\begin{verbatim}
command: : abs        Input-signal1, Input-signal2,  Output-signal
default  :                 a       ,      b       ,   input-signal2
range    :               <a-z>     ,    <a-z>     ,      <a-z>
domain   : time, frequency, amplitude, magnitude, phase
\end{verbatim}
\ec

\paragraph{add}: Add two signals

\[ Re(out) = Re(in1) + Re(in2) \]
\[ Im(out) = Im(in1) + Im(in2) \]

\bc
\begin{verbatim}
command: : add        Input-signal1, Input-signal2,  Output-signal
default  :                 a       ,      b       ,   input-signal2
range    :               <a-z>     ,    <a-z>     ,      <a-z>
domain   : time, frequency, amplitude, magnitude, phase
\end{verbatim}
\ec

\paragraph{substract}: Substract two signals

\[ Re(out) = Re(in1) - Re(in2) \]
\[ Im(out) = Im(in1) - Im(in2) \]

\bc
\begin{verbatim}
command: : substract  Input-signal1, Input-signal2,  Output-signal
default  :                 a       ,      b       ,   input-signal2
range    :               <a-z>     ,    <a-z>     ,      <a-z>
domain   : time, frequency, amplitude, magnitude, phase
\end{verbatim}
\ec

\paragraph{multiply}: Multiply two signals

\[ Re(out) = \frac{ Re(in1) * Re(in2) - Im(in1) * Im(in2) }{C} \]
\[ Im(out) = \frac{ Re(in1) * Im(in2) + Im(in1) * Re(in2) }{C} \]

\bc
\begin{verbatim}
command: : multiply   Input-signal1, Input-signal2,  constant, Output-signal
default  :                 a       ,      b       ,      1   ,  input-signal2
range    :               <a-z>     ,    <a-z>     ,   <1-max>,     <a-z>
domain   : time, frequency, amplitude, magnitude, phase
\end{verbatim}
\ec

\paragraph{divide}: Divide two signals

\[ Re(out) = \frac{\sqrt{ Re^{2}(in1) + Im^{2}(in1)} }{ 1 + \sqrt{Re^{2}(in2) + Im^{2}(in2)}} \]
\[ Im(out) = 0 \]

\bc
\begin{verbatim}
command: : divide     Input-signal1, Input-signal2,  Output-signal
default  :                 a       ,      b       ,   input-signal2
range    :               <a-z>     ,    <a-z>     ,      <a-z>
domain   : time, frequency, amplitude, magnitude, phase
\end{verbatim}
\ec

\paragraph{sine}: Calculate sine

\[ Re(out) = \sin (Re(in)) \]
\[ Im(out) = 0 \]

\bc
\begin{verbatim}
command: : sine       Input-signal,  Output-signal
default  :                 a      ,  input-signal 
range    :               <a-z>    ,     <a-z>
domain   : time, frequency, amplitude, magnitude, phase
\end{verbatim}
\ec

\paragraph{cosine}: Calculate sine

\[ Re(out) = \cos (Re(in)) \]
\[ Im(out) = 0 \]

\bc
\begin{verbatim}
command: : cosine     Input-signal,  Output-signal
default  :                 a      ,  input-signal 
range    :               <a-z>    ,     <a-z>
domain   : time, frequency, amplitude, magnitude, phase
\end{verbatim}
\ec

\paragraph{ln}: logarithm

\[ Re(out) = \ln{Re(in)} \]
\[ Im(out) = 0 \]

\bc
\begin{verbatim}
command: : ln         Input-signal,  Output-signal
default  :                 a      ,  input-signal 
range    :               <a-z>    ,     <a-z>
domain   : time, frequency, amplitude, magnitude, phase
\end{verbatim}
\ec

\paragraph{log}: $ ^{10}\log $

\[ Re(out) = \log{Re(in)} \]
\[ Im(out) = 0 \]

\bc
\begin{verbatim}
command: : log        Input-signal,  Output-signal
default  :                 a      ,  input-signal 
range    :               <a-z>    ,     <a-z>
domain   : time, frequency, amplitude, magnitude, phase
\end{verbatim}
\ec

\paragraph{epow}: 

\[ Re(out) = e^{Re(in)} \]
\[ Im(out) = 0 \]

\bc
\begin{verbatim}
command: : epow       Input-signal,  Output-signal
default  :                 a      ,  input-signal 
range    :               <a-z>    ,     <a-z>
domain   : time, frequency, amplitude, magnitude, phase
\end{verbatim}
\ec

\paragraph{tenpow}: 

\[ Re(out) = 10^{Re(in)} \]
\[ Im(out) = 0 \]

\bc
\begin{verbatim}
command: : tenpow     Input-signal,  Output-signal
default  :                 a      ,  input-signal 
range    :               <a-z>    ,     <a-z>
domain   : time, frequency, amplitude, magnitude, phase
\end{verbatim}
\ec


\paragraph{minimum}: minimum of two signals

\[ Re(out) = \sqrt{Re^{2}(in1) + Im^{2}(in1)} < \sqrt{Re^{2}(in2) + Im^{2}(in2)} ? Re(in1) : Re(in2) \]
\[ Im(out) = \sqrt{Re^{2}(in1) + Im^{2}(in1)} < \sqrt{Re^{2}(in2) + Im^{2}(in2)} ? Im(in1) : Im(in2) \]

\bc
\begin{verbatim}
command: : minimum    Input-signal1, Input-signal2,  Output-signal
default  :                 a       ,      b       ,  input-signal2
range    :               <a-z>     ,    <a-z>     ,      <a-z>
domain   : time, frequency, amplitude, magnitude, phase
\end{verbatim}
\ec

\paragraph{maximum}: maximum of two signals

\[ Re(out) = \sqrt{Re^{2}(in1) + Im^{2}(in1)} > \sqrt{Re^{2}(in2) + Im^{2}(in2)} ? Re(in1) : Re(in2) \]
\[ Im(out) = \sqrt{Re^{2}(in1) + Im^{2}(in1)} > \sqrt{Re^{2}(in2) + Im^{2}(in2)} ? Im(in1) : Im(in2) \]

\bc
\begin{verbatim}
command: : maximum    Input-signal1, Input-signal2,  Output-signal
default  :                 a       ,      b       ,  input-signal2
range    :               <a-z>     ,    <a-z>     ,      <a-z>
domain   : time, frequency, amplitude, magnitude, phase
\end{verbatim}
\ec

\section{Cross-correlation}

The cross-correlation of two signals of N samples is calculated as follows:

\begin{enumerate}
  \item Zeropad the signals per record, to prevent wrap-around pollution.
  \item Multiply the signals with an N points window.
  \item Calculate the 2N points FFT of the signals.
  \item Multiply the complex conjugate of the first with the second signal.
  \item Calculate the 2N points IFFT.
\end{enumerate}

\paragraph{correlation}: calculate cross-correlation of two signals

\bc
\begin{verbatim}
command: : correlation  Input-signal1, Input-signal2, Output-signal, window-type
default  :                   a       ,      b       , input-signal2,      0
range    :                 <a-z>     ,    <a-z>     ,    <a-z>     ,    <0-6>
domain   : time
\end{verbatim}
\ec

\begin{verbatim}
Windows:
0  block
1  Hanning
2  Hamming
3  Gauss
4  Blackman
5  Kaiser
6  triangle
\end{verbatim}

\section{Convolution}

The convolution of two signals of N samples is calculated as follows:

\begin{enumerate}
  \item Zeropad the signals per record, to prevent wrap-around pollution.
  \item Multiply the signals with an N points window.
  \item Calculate the 2N points FFT of the signals.
  \item Multiply the first with the second signal.
  \item Calculate the 2N points IFFT.
\end{enumerate}


\paragraph{convolution}: calculate convolution of two signals

\bc
\begin{verbatim}
command: : convolution  Input-signal1, Input-signal2, Output-signal, window-type
default  :                   a       ,      b       , input-signal2,      0
range    :                 <a-z>     ,    <a-z>     ,    <a-z>     ,    <0-6>
domain   : time
\end{verbatim}
\ec

\section{Functions}

ASM can generate some standard signals.
Currently the functions are limited to 1 record and 1 channel.

\begin{itemize}
\item ts = sample time
\item A = Amplitude
\item Amax = 2147483647
\item B = offset
\item Bmin = -2147483647
\item Bmax = 2147483647
\item T = period time
\item f = frequency
\item fmax = 2147483647
\item data-type: 0=real, 1=imaginary, 2=complex
\item N = number of elements = $2^{n}$
Nmin = 128 (n=7), Nmax = 4096
\item Smax = 65535, maximal sample-rate (in 10 Hz)
\end{itemize}

Sample-rate is always given in 10 (Hz).

\paragraph{fdelta}:

\[ x(n \cdot t_{s}) = A,~ n \cdot t_{s} = t_{d},~ n=0,1,...,N-1 \]
\[ x(n \cdot t_{s}) = 0,~ n \cdot t_{s} \neq t_{d},~ n=0,1,...,N-1 \]

\bc
\begin{verbatim}
command: : fdelta  signal,     A    ,    td (ms) , data-type,    n   , sample-rate
default  :           a   ,    255   ,     0      ,     0    ,    9   ,   1024
range    :         <a-z> , <0-Amax> , <0-t(N-1)> ,   <0-2>  ,  <7-12>,  <1-Smax>
domain   : time
\end{verbatim}
\ec

\paragraph{fconstant}:

\[ x(n \cdot t_{s}) = A,~ n=0,1,...,N-1 \]
\bc
\begin{verbatim}
command: : fconstant  signal,     A    , data-type,    n   , sample-rate
default  :              a   ,    255   ,     0    ,    9   ,   1024
range    :            <a-z> , <0-Amax> ,   <0-2>  ,  <7-12>,  <1-Smax>
domain   : time
\end{verbatim}
\ec

\paragraph{fstep}:

\[ x(n \cdot t_{s}) = B,~ n \cdot t_{s} < t_{step},~ n=0,1,...,N-1 \]
\[ x(n \cdot t_{s}) = A+B,~ n \cdot t_{s} \geq t_{step},~ n=0,1,...,N-1 \]

\bc
\begin{verbatim}
command: : fstep  signal,      B     ,     A    , tdelay(ms) , data-type,    n   , sample-rate
default  :          a   ,      0     ,    255   ,     0      ,     0    ,    9   ,   1024
range    :        <a-z> , <Bmin-Bmax>, <0-Amax> , <0-t(N-1)> ,   <0-2>  ,  <7-12>,  <1-Smax>
domain   : time
\end{verbatim}
\ec

\paragraph{fsquare}:

\[ x(n \cdot t_{s}) = A+B,~ 0 \leq n \cdot t_{s} < dc \cdot T,~ n=0,1,...,N-1 \]
\[ x(n \cdot t_{s}) = -A+B,~ dc \cdot T \leq n \cdot t_{s} < T,~ n=0,1,...,N-1 \]

\bc
\begin{verbatim}
command: : fsquare  signal,      B     ,     A    ,     f    ,   dc   , data-type,    n   , sample-rate
default  :            a   ,      0     ,    255   ,    100   ,   50   ,     0    ,    9   ,   1024
range    :          <a-z> , <Bmin-Bmax>, <0-Amax> , <1-fmax> , <0-100>,   <0-2>  ,  <7-12>,  <1-Smax>
domain   : time
\end{verbatim}
\ec

dc = duty cycle

\paragraph{framp}:

\[ x(n \cdot t_{s}) = \frac{A}{t_{s} \cdot (N-1)} \cdot n \cdot t_{s} + B,~ n=0,1,...,N-1 \]

\bc
\begin{verbatim}
command: : framp  signal,      B     ,     A    ,data-type,    n   , sample-rate
default  :          a   ,      0     ,    255   ,   0     ,    9   ,   1024
range    :        <a-z> , <Bmin-Bmax>, <0-Amax> ,  <0-2>  ,  <7-12>,  <1-Smax>
domain   : time
\end{verbatim}
\ec

\paragraph{ftriangle}:

\[ x(n \cdot t_{s}) = \frac{4A}{T}\cdot(n \cdot t_{s} + \frac{\phi_{0} \cdot T}{2\pi})-A+B,~ 0 \leq n \cdot t_{s} < \frac{1}{2}T,~ n=0,1,...,N-1 \]
\[ x(n \cdot t_{s}) = \frac{-4A}{T}\cdot(n \cdot t_{s} + \frac{\phi_{0} \cdot T}{2\pi})+3A+B,~ \frac{1}{2}T \leq n \cdot t_{s} < T,~ n=0,1,...,N-1 \]

\bc
\begin{verbatim}
command: : ftriangle  signal,      B     ,     A    ,     f    ,   phi0 , data-type,    n   , sample-rate
default  :              a   ,      0     ,    255   ,    100   ,    0   ,       0  ,    9   ,   1024
range    :            <a-z> , <Bmin-Bmax>, <0-Amax> , <1-fmax> , <0-2pi>,   <0-2>  ,  <7-12>,  <1-Smax>
domain   : time
\end{verbatim}
\ec

\paragraph{fsine}:

\[ x(n \cdot t_{s}) = A \sin(2\pi fnt_{s} + \phi_{0}) + B,~ n=0,1,...,N-1 \]

\bc
\begin{verbatim}
command: : fsine  signal,      B     ,     A    ,     f    ,   phi0 , data-type,    n   , sample-rate
default  :          a   ,      0     ,    255   ,    100   ,    0   ,   0      ,    9   ,   1024
range    :        <a-z> , <Bmin-Bmax>, <0-Amax> , <1-fmax> , <0-2pi>,   <0-2>  ,  <7-12>,  <1-Smax>
domain   : time
\end{verbatim}
\ec

\paragraph{fsinc}:

\[ x(n \cdot t_{s}) = A \left( \frac{\sin (2\pi fnt_{s})}{2\pi fnt_{s}} \right) + B,~ n=1,...,N-1 \]
\[ x(n \cdot t_{s}) = A \cos (2\pi fnt_{s})+B,~ n=0 \]

\bc
\begin{verbatim}
command: : fsinc  signal,      B     ,     A    ,     f    , data-type,    n   , sample-rate
default  :          a   ,      0     ,    255   ,    100   ,   0      ,    9   ,   1024
range    :        <a-z> , <Bmin-Bmax>, <0-Amax> , <1-fmax> ,   <0-2>  ,  <7-12>,  <1-Smax>
domain   : time
\end{verbatim}
\ec

\paragraph{fcosine}:

\[ x(n \cdot t_{s}) = A \cos (2\pi fnt_{s} + \phi_{0})+B,~ n=0,...,N-1 \]

\bc
\begin{verbatim}
command: : fcosine  signal,      B     ,     A    ,     f    ,   phi0 , data-type,    n   , sample-rate
default  :            a   ,      0     ,    255   ,    100   ,    0   ,   0      ,    9   ,   1024
range    :          <a-z> , <Bmin-Bmax>, <0-Amax> , <1-fmax> , <0-2pi>,   <0-2>  ,  <7-12>,  <1-Smax>
domain   : time
\end{verbatim}
\ec

\paragraph{fexp}:

\[  x(n \cdot t_{s}) = A ( 1 - e^{\frac{-n\cdot t_{s}}{t_{63.2\%}}}),~ n=0,...,N-1 \]

\bc
\begin{verbatim}
command: : fexp     signal,     A    , t63.2(ms), data-type,    n   , sample-rate
default  :            a   ,    255   ,    0.10  ,     0    ,    9   ,       1024
range    :          <a-z> , <0-Amax> ,<1e-6-1e6>,   <0-2>  ,  <7-12>,  <1-Smax>
domain   : time
\end{verbatim}
\ec

\paragraph{fnoise}: pseudo random noise.

\bc
\begin{verbatim}
command: : fnoise  signal,     A    , data-type,  seed  ,   n   , sample-rate
default  :           a   ,    255   ,     0    ,    1   ,   9   ,    1024
range    :         <a-z> , <0-Amax> ,   <0-2>  , <0-512>, <7-12>,  <1-Smax>
domain   : time
\end{verbatim}
\ec

\section{Conditioning}

Zero padding is extending each record of N samples with N zeros.
The output signal has a record length of 2N.

\paragraph{zeropadding}:

\bc
\begin{verbatim}
command: : zeropad  input-signal, output-signal
default  :               a      ,  input-signal
range    :             <a-z>    ,     <a-z>
domain   : time
\end{verbatim}
\ec

\section{Windowing}

\paragraph{wblock}: block window

\[ W(n) = 1,~ n = 0,1,...,N-1 \]

\bc
\begin{verbatim}
command: : wblock   signal,    n    , samplerate
default  :            a   ,    9    ,    1024
range    :          <a-z> ,  <7-12> ,  (1-Smax)
domain   : time
\end{verbatim}
\ec

\paragraph{whanning}: Hanning window

\[ W(n) = \frac{1}{2}\left( 1- \cos (\frac{2\pi n}{N-1}) \right) ,~ n = 0,1,...,N-1 \]

\bc
\begin{verbatim}
command: : whanning   signal,    n    , samplerate
default  :              a   ,    9    ,    1024
range    :            <a-z> ,  <7-12> ,  (1-Smax)
domain   : time
\end{verbatim}
\ec

\paragraph{whamming}: Hamming window

\[W(n) =  0.538 - 0.462 \cos (\frac{2\pi n}{N-1}) ,~ n = 0,1,...,N-1 \]

\bc
\begin{verbatim}
command: : whamming   signal,    n    , samplerate
default  :              a   ,    9    ,    1024
range    :            <a-z> ,  <7-12> ,  (1-Smax)
domain   : time
\end{verbatim}
\ec

\paragraph{wgauss}: Gauss window
\large
\[W(n) = e^{- \frac{1}{2} \left( \frac{\alpha \cdot (n - \frac{N-1}{2}) \cdot 2 }{\frac{N-1}{2}} \right)^{2}},~ n = 0,1,...,N-1 \]
\normalsize
\[ \alpha = 3.0 \]

\bc
\begin{verbatim}
command: : wgauss     signal,    n    , samplerate
default  :              a   ,    9    ,    1024
range    :            <a-z> ,  <7-12> ,  (1-Smax)
domain   : time
\end{verbatim}
\ec

\paragraph{wblackman}: Blackman window

\[ W(n) = 0.42 - 0.5\cos (\frac{2\pi n}{N-1}) + 0.08\cos (\frac{4\pi n}{N-1}),~ n = 0,1,...,N-1 \]

\bc
\begin{verbatim}
command: : wblackman  signal,    n    , samplerate
default  :              a   ,    9    ,    1024
range    :            <a-z> ,  <7-12> ,  (1-Smax)
domain   : time
\end{verbatim}
\ec

\paragraph{wkaiser}: Kaiser window

\[ W(n) = \frac{1}{2.48} \left( 1- 1.24\cos (\frac{2\pi n}{N-1}) + 0.244\cos (\frac{4\pi n}{N-1}) - 0.00305\cos (\frac{6\pi n}{N-1}) \right),\]
\[ n = 0,1,...,N-1  \]


\bc
\begin{verbatim}
command: : wkaiser    signal,    n    , samplerate
default  :              a   ,    9    ,    1024
range    :            <a-z> ,  <7-12> ,  (1-Smax)
domain   : time
\end{verbatim}
\ec

\paragraph{wtriangle}:

\[ W(n) = 1 - \left|  \frac{n-\frac{N-1}{2}}{\frac{N-1}{2}} \right| ,~ n = 0,1,...,N-1 \]

\bc
\begin{verbatim}
command: : wtriangle  signal,    n    , samplerate
default  :              a   ,    9    ,    1024
range    :            <a-z> ,  <7-12> ,  (1-Smax)
domain   : time
\end{verbatim}
\ec

\section{Presentation functions}

\subsection{Time domain}\label{sec:timedom}

\paragraph{real}: Show the real part of the signal.

\bc
\begin{verbatim}
command: : real    signal, channel-no ,  record-no
default  :           a   ,      0     ,      0
range    :         <a-z> ,  <0-max>   ,   <0-max>
domain   : time, frequency
\end{verbatim}
\ec


\paragraph{imaginary}: Show the imaginary part of the signal.

\bc
\begin{verbatim}
command: : imaginary  signal, channel-no ,  record-no
default  :              a   ,      0     ,      0
range    :             <a-z> ,  <0-max>   ,   <0-max>
domain   : time, frequency
\end{verbatim}
\ec

\subsection{Frequency domain}

\paragraph{real}: See section \ref{sec:timedom}

\paragraph{imaginary}: See section \ref{sec:timedom}

\paragraph{bode}: Show bode diagram

\bc
\begin{verbatim}
command: : bode       signal, channel-no ,  record-no
default  :              a   ,      0     ,      0
range    :            <a-z> ,  <0-max>   ,   <0-max>
domain   : frequency
\end{verbatim}
\ec

\subsection{Magnitude and phase domain}

\subsection{Generic functions}

\paragraph{doff}: display off, do not display signals

\bc
\begin{verbatim}
command: : doff
\end{verbatim}
\ec

\paragraph{don}: display on, display signals

\bc
\begin{verbatim}
command: : don
\end{verbatim}
\ec

\paragraph{boff}: bar display off, display signals as normal graphs.

\bc
\begin{verbatim}
command: : boff
\end{verbatim}
\ec

\paragraph{bon}: bar display on, display signals as bargraphs.

\bc
\begin{verbatim}
command: : bon
\end{verbatim}
\ec

\paragraph{display}: display a signal, regardless of \textbf{doff/don}.

\bc
\begin{verbatim}
command: : display    signal, channel-no ,  record-no
default  :              a   ,      0     ,      0
range    :            <a-z> ,  <0-max>   ,   <0-max>
domain   : time, frequency, magnitude, phase, amplitude
\end{verbatim}
\ec

\paragraph{xscale}: set horizontal scale factor.

\bc
\begin{verbatim}
command: : xscale     signal, scale-factor
default  :              a   ,      1
range    :            <a-z> ,    <0-10>
domain   : time, frequency, magnitude, phase, amplitude
\end{verbatim}
\ec

\paragraph{print}: print values of signal in commandline window.

\bc
\begin{verbatim}
command: : print      signal, channel-no ,  record-no
default  :              a   ,      0     ,      0
range    :            <a-z> ,  <0-max>   ,   <0-max>
domain   : time, frequency, magnitude, phase, amplitude
\end{verbatim}
\ec

\paragraph{info}: print header information.

\bc
\begin{verbatim}
command: : info      signal
default  :              a   
range    :            <a-z> 
domain   : time, frequency, magnitude, phase, amplitude
\end{verbatim}
\ec

\paragraph{list}: list all signals.

\bc
\begin{verbatim}
command: :  list
\end{verbatim}
\ec

\section{Other functions}

\paragraph{writf}: write file.

\bc
\begin{verbatim}
command: : writef    signal,  filename  , usertext, description, bits-per-sample
default  :              a  ,   a.asm    ,         ,            , bits-per-sample
domain   : time, frequency, magnitude, phase, amplitude
\end{verbatim}
\ec

\paragraph{readf}: read file.

\bc
\begin{verbatim}
command: : readf   filename  , signalname
default  :           a.asm   , signalname-in-file
domain   : time, frequency, magnitude, phase, amplitude
\end{verbatim}
\ec

\paragraph{copy}: Copy a signal.

\bc
\begin{verbatim}
command: : copy       Input-signal,  Output-signal
default  :                 a      ,      copy
range    :               <a-z>    ,      <a-z>
domain   : time, frequency, amplitude, magnitude, phase
\end{verbatim}
\ec

\paragraph{rename}: Rename a signal.

\bc
\begin{verbatim}
command: : Rename     Input-signal,  Output-signal
default  :                 a      ,        a
range    :               <a-z>    ,      <a-z>
domain   : time, frequency, amplitude, magnitude, phase
\end{verbatim}
\ec


\paragraph{shift}: shift a signal.

\bc
\begin{verbatim}
command: : shift     signal ,     shift      , output-signal
default  :              a   ,       0        ,  input-signal
range    :            <a-z> ,<-length-length>,     <a-z>
domain   : time, frequency, magnitude, phase, amplitude
\end{verbatim}
\ec

Shifting is done per channel.
Maximum shift is one record length.

\paragraph{rotate}: rotate a signal.

\bc
\begin{verbatim}
command: : rotate    signal ,    rotate      , output-signal
default  :              a   ,       0        ,  input-signal
range    :            <a-z> ,<-length-length>,     <a-z>
domain   : time, frequency, magnitude, phase, amplitude
\end{verbatim}
\ec

Rotation is done per channel.
Maximum rotation is one record length.

\paragraph{clip}: clip a signal.

\bc
\begin{verbatim}
command: : clip      signal ,     left       ,   right   , output-signal
default  :              a   ,       0        ,  length   ,  input-signal
range    :            <a-z> ,   <0-length>   , <0-length>,     <a-z>
domain   : time
\end{verbatim}
\ec

\bc
\begin{verbatim}
command: : clip      signal ,     left       ,   right   , attenuation ,output-signal
default  :              a   ,       0        ,  length   ,    50 (dB)  , input-signal
range    :            <a-z> ,   <0-1/2Fs>    , <0-1/2Fs> ,    <0-100>  ,    <a-z>
domain   : frequency
\end{verbatim}
\ec

Clipping is done per record.


\section{New functions}

The functions listed in this section are new. They did not exist in
the original RISC Os version of ASM.

\paragraph{signaldir}: Set the directory path where to store and read signals.

\bc
\begin{verbatim}
command: signaldir  directory-path
\end{verbatim}
\ec


\chapter{Data format}

\section{Header}\label{sec:header}

All signals have information in the form of a header. To show header
information use the \textbf{info} command. The ASM header is a
TCL-Image header with some additions. The ASM header is defined as follows:

\paragraph{}
In this case a word is two bytes (16 bits). The header has a
fixed size of 512 bytes.

\begin{description}
\item[Word(s)] Contents
\item[1] Unused and always 0 in ASM.
         \begin{description}
             \item[\sf 0] = if file contains 16 bits pixels, `unpacked'.
             \item[\sf 1] = if file contains 8 bits pixels, `packed'.
	     \end{description}
\item[2] Number of samples of the record. min $2^{7}=128$. max $2^{12}=4096$.
\item[3] Number of channels. min 1. max 65535.
\item[4] File sequence number on tape, starting with 1. On disk this is
             always 0.
\item[5] Number of bits per sample.
         \begin{description}
             \item[\sf 8]    = 8-bits amplitude (byte)
			 \item[\sf 16]   = 16-bits amplitude (short)
			 \item[\sf 32]   = 32-bits amplitude (integer)
			 \item[\sf 3232] = 32-bits amplitude (float)
			 \item[\sf 6464] = 64-bits amplitude (double)
	     \end{description}
\item[6] Number of records per channel. min 1. max 65535.
\item[7] Domain ID
         \begin{description}
             \item[\sf 0]   = Time
			 \item[\sf 1]   = Frequency
			 \item[\sf 2]   = Amplitude
			 \item[\sf 3]   = Magnitude
			 \item[\sf 4]   = Phase
	     \end{description}
\item[8] Data Type ID
         \begin{description}
             \item[\sf 0]   = Real
			 \item[\sf 1]   = Imaginary
			 \item[\sf 2]   = Complex
	     \end{description}
\item[9-10] Pointer to real part
\item[11-12] Pointer to imaginary part
\item[13] Samplerate in $10^{1} (Hz)$. max 65535.
\item[14-32] Reserved
\item[33-128] Numeric data (type double)
\item[129-165] ASM ID string (ASCII text)
\item[166-181] Signalname (ASCII text)
\item[182-204] User text (ASCII text)
\item[205-219] Date (ASCII text)
\item[220-256] Description (ASCII text)
\end{description}

\section{Data}

\paragraph{channels}
A signal can exist out of one or more channels. Channels
are in time parallel recordings of sound. This could be
done with multiple microphones. Every microphone records
a channel. A channel exists out of one or more records.
All channels have the same number of records. The maximum
number of records is 65535.

\paragraph{records}
A channel is divided in records of equal size. The size is always
a power of 2. The minimal size is $2^{7}=128$, and the
maximal size is $2^{12}=4096$ samples. This is done to
keep the size of the data manageable. Fourier transformation is
done per record. Also displaying the signal is done per record.

\paragraph{samples}
Samples can be real, imaginary or complex. In memory samples
are always of type double (64 bit floating point). Samples
can be converted to lower resolutions when writing them to disk.
While reading data from disk ASM will convert the samples to
type double.

\paragraph{}
The functions in ASM generate signals that exist out of one channel
with one record.

\begin{figure}[h]
\centerline{\includegraphics{asmdata}}
\caption{ASM file}
\label{fig:asmfile}
\end{figure}



\end{document}               % End of document.
